%%%%%%%%%%%%%%%%%%%%%%%%%%%%%%%%%%%%%%%%%%%%%%%%%%%%%%%%%%%
%
%    Exam Package for plainTeX
%    Run this only with LuaTeX
%    Author : Sungbae Jeong (Almagest)
%    Release Date : 21/01/19
%    Version : 3.00
%
%%%%%%%%%%%%%%%%%%%%%%%%%%%%%%%%%%%%%%%%%%%%%%%%%%%%%%%%%%%

% Check whether the engine is luatex
% If not, then panic it
\expandafter\ifx\csname luatexversion\endcsname\else
\errmessage{Use LuaTeX to compile this pacage}
\fi

%%%%%%%%%%%%%%%%%%%%%%%%
%  Importing Packages  %
%%%%%%%%%%%%%%%%%%%%%%%%
\input luaotfload.sty
\input kotexplain
\input ams-math
\input eplain
\input opmac
\input tikz
\usetikzlibrary{intersections,calc,arrows.meta,matrix}

%%%%%%%%%%%%%%%%%%%%%%%%%%%%%%%%%%%%%%%%%%%%%%%%%%%%%%%%%%%
%
%    Starting Implementation
%
%%%%%%%%%%%%%%%%%%%%%%%%%%%%%%%%%%%%%%%%%%%%%%%%%%%%%%%%%%%
\catcode`@=11

%%%%%%%%%%%%%%%%%%%%%%%%
%  Define Basic Fonts  %
%%%%%%%%%%%%%%%%%%%%%%%%
\font\smallmathtext=cmsy7 at 3.5pt
\newfam\smallfam \textfont\smallfam=\smallmathtext \scriptfont\smallfam=\smallmathtext
\scriptscriptfont\smallfam=\smallmathtext
\def\smallmath{\fam\smallfam}
\font\tenamsb=msbm10 \font\sevenamsb=msbm7 \font\fiveamsb=msbm5
\newfam\bbfam \textfont\bbfam=\tenamsb \scriptfont\bbfam=\sevenamsb \scriptscriptfont\bbfam=\fiveamsb
\def\bbb{\fam\bbfam}
\let\scr=\script

\hangulpunctuations=0
\hangulfont="Noto Serif KR" at 10pt
\hanjafont="Noto Serif KR" at 10pt
\font\titlefont="Noto Sans KR Bold" at 21pt
\font\titlewarningfont="Noto Sans KR Bold" at 18pt
\font\subtitlefont="Noto Sans KR Bold" at 15pt
\font\titleexplainfont="Noto Sans KR" at 10pt
\font\mainfont="Noto Serif KR" at 10pt
\font\headfont="Noto Sans KR Bold" at 10pt
\font\footfont="Noto Sans KR Bold" at 10pt
\font\boldfont="Noto Serif KR Bold" at 10pt
\font\sansfont="Noto Sans KR" at 10pt
\font\bsansfont="Noto Sans KR Bold" at 10pt
\font\explainfont="Noto Serif KR" at 8.5pt
\font\expempfont="Noto Serif KR Bold" at 8.5pt

\let\oldbf\bf
\def\bf{\oldbf\boldfont}
\def\sf{\sansfont}
\def\bsf{\bsansfont}

%%%%%%%%%%%%%%%%%%%%%%%%%%%%%%%%%%%%%%%%%%%
%  Defining Basic Commands and Variables  %
%%%%%%%%%%%%%%%%%%%%%%%%%%%%%%%%%%%%%%%%%%%
% Define Image relative code inside Luatex
\let\pdfximage\saveimageresource
\let\pdflastximage\lastsavedimageresourceindex
\let\pdflastximagepages\lastsavedimageresourcepages
\let\pdfrefximage\useimageresource

% Total Score constant in lua code
\directlua{scorecnt = 0}

% Constants (Include Boolean)
\newtoks\maintitle \newtoks\subtitle
\newcount\timelimit
\newread\sc@reread \newwrite\sc@rewrite
\newif\ifish@mew@rk \newif\ifsubpr@blemused \newif\ifanswershown \newif\ifansdisplaysh@on
\newbox\titleb@x \newbox\leftb@x \newbox\headb@xeven \newbox\headb@xodd
\newdimen\@vsize \newdimen\@hsize
\newcount\questi@ncnt \newcount\subquesti@ncnt
\let\showanswer=\answershowntrue

% Boxed
\def\boxit#1#2#3{\vbox{\hrule height#1\hbox to\hsize{\vrule width#1\kern#2pt\hss\vbox{\vskip#2pt #3 \vskip#2pt}%
\hss\kern#2pt\vrule width#1}\hrule height#1}}

% Question and Score Command
\outer\def\question{\ifsubpr@blemused \subquesti@ncnt=0\fi
\subpr@blemusedfalse
\par\vfill\global\advance\questi@ncnt by 1{\bf\the\questi@ncnt.}\enskip\ignorespaces}
\outer\def\subquestion{\ifsubpr@blemused\par\vfill\else\par\vskip5pt\fi
\subpr@blemusedtrue
\global\advance\subquesti@ncnt by 1{(\the\subquesti@ncnt)}\enskip\ignorespaces}
\outer\def\score#1;{{\directlua{scorecnt = scorecnt + #1}
\unskip\nobreak\hfil\penalty50\hskip2em\hbox{}\nobreak\hfil[#1점]
\parfillskip=0pt \finalhyphendemerits=0 \par}}
\outer\def\Score#1;{{\unskip\nobreak\hfil\penalty50\hskip2em\hbox{}\nobreak\hfil[#1점]
\parfillskip=0pt \finalhyphendemerits=0 \par}}

\outer\def\beginfigure{\par\vskip5pt\hbox to\hsize\bgroup\hss}
\outer\def\endfigure{\hss\egroup}

\let\q=\question
\let\sq=\subquestion

% ExplainBox
\def\beginexplain{\par\vskip5pt\hbox to\hsize\bgroup\hss
\vbox\bgroup\hsize=.9\hsize\hrule\hbox\bgroup\vrule\kern5pt\vbox\bgroup\kern5pt}
\def\endexplain{\par\kern5pt\egroup\kern5pt\vrule\egroup\hrule\egroup\hss\egroup}
\let\explain\beginexplain

% Numbering
\def\cn#1{\ifnum #1<1
\hbox{\ooalign{$\mathchar"020D$\cr\hidewidth\lower.1ex\hbox{$?$}\hidewidth\cr}}
\else\ifnum #1<10
\hbox{\ooalign{$\mathchar"020D$\cr\hidewidth\lower.1ex\hbox{$#1$}\hidewidth\cr}}
\else
\hbox{\ooalign{$\mathchar"020D$\cr\hidewidth\lower.1ex\hbox{$?$}\hidewidth\cr}}
\fi\fi}
\let\binomial=\choose
\outer\def\choose#1{\par\nobreak\kern5pt\nobreak\vbox{\halign to\hsize{&\tabskip=0pt\relax
\cn{##}\kern3pt&##\hfil\tabskip=0pt plus1fil\crcr#1\crcr}}}
\outer\def\mchoose#1{\par\nobreak\kern5pt\nobreak\vbox{\halign to\hsize{&\tabskip=0pt\relax
\cn{##}\kern3pt&$##$\hfil\tabskip=0pt plus1fil\crcr#1\crcr}}}
\outer\def\dchoose#1{\par\nobreak\kern5pt\nobreak\vbox{\halign to\hsize{&\tabskip=0pt\relax
\cn{##}\kern3pt&$\displaystyle##$\hfil\tabskip=0pt plus1fil\crcr#1\crcr}}}

% Korean Itemize
\def\koenum#1{\ifcase#1 (???)\or%
(가)\or(나)\or(다)\or(라)\or(마)\or(바)\or(사)\or(아)\or%
(자)\or(차)\or(카)\or(타)\or(파)\or(하)\else(???)\fi\enskip\ignorespaces}

%%%%%%%%%%%%%%%%%%%%%
%  Defining Colors  %
%%%%%%%%%%%%%%%%%%%%%
\protected\def\pdfcolorstack{\pdfextension colorstack}
\def\pdfcolorstackinit{\pdffeedback colorstackinit}
\mathchardef\colorcnt=\pdfcolorstackinit page {0 g 0 G}
\def\colorpop{\pdfcolorstack\colorcnt pop}
\def\colorpush#1{\pdfcolorstack\colorcnt push {#1 k #1 K}}
\def\colorset#1{\pdfcolorstack\colorcnt set {#1 k #1 K}}
\def\answercolor{\colorpush{0 1 1 0}\aftergroup\colorpop}
\def\answer#1{\ifanswershown\par\null\hfill{\answercolor\bf Ans : #1}\fi}

%%%%%%%%%%%%%%%%%%%%%%%%%%%%%%%%%%%%
%  Defining Math Related Commands  %
%%%%%%%%%%%%%%%%%%%%%%%%%%%%%%%%%%%%
% Math Floating Point Command
\def\fr#1{\begingroup
\setbox0=\hbox{#1}
\advance\dimen0 by \ht0
\advance\dimen0 by .6pt
\ooalign{\raise\dimen0\hbox to\wd0{\hss$\mathchar"0B0F$\hss}\cr$#1$\cr}
\endgroup}
\let\emptyset\varnothing

% Sets
\def\ALPHABETList{A,B,C,D,E,F,G,H,I,J,K,L,M,N,O,P,Q,R,S,T,U,V,W,X,Y,Z}
\for\ALPHABET:=\ALPHABETList\do{%
    \expandafter\xdef\csname\ALPHABET f\endcsname{{\frak\ALPHABET}}
    \expandafter\xdef\csname\ALPHABET c\endcsname{{\cal\ALPHABET}}
    \expandafter\xdef\csname\ALPHABET b\endcsname{{\bbb\ALPHABET}}
    \expandafter\xdef\csname\ALPHABET s\endcsname{{\scr\ALPHABET}}
}
\let\N=\Nb \let\Z=\Zb \let\Q=\Qb \let\R=\Rb \let\C=\Cb

%%%%%%%%%%%%%%%%%%%%%%%%%%%%%%%%
%  Switch of type of the exam  %
%%%%%%%%%%%%%%%%%%%%%%%%%%%%%%%%
\newif\iftwocols

% Default exam type is twocols
\twocolstrue

\let\onecol\twocolsfalse
\let\twocol\twocolstrue
\let\twocols\twocolstrue

\outer\def\loadexam{%
}

\outer\def\beginexam{%
\openin\sc@reread=\jobname.tdt\relax
\ifeof\sc@reread
\let\t@talsc@re\relax
\let\t@talpagecnt=0
\else
\read\sc@reread to\t@talsc@re
\read\sc@reread to\t@talpagecnt
\fi
\closein\sc@reread
\iftwocols
    \def\breakcolumn{\vfill\eject}
    \def\nextq{\vfill\eject\hbox to\hsize{}\vfill\eject}
    \let\bc\breakcolumn
    % PageNumber
    \def\pagenum{\tikz{%
    \path (11.5pt,15.3pt) node {\the\pageno};
    \path (34.5pt,7.67pt) node {\t@talpagecnt};
    \draw[thick] (0,0) rectangle (46pt,23pt);
    \draw[thick] (0,0) -- (46pt,23pt);
    }}
    \let\leftright=L \@vsize=\vsize \@hsize=1.1\hsize \hsize=\@hsize
    \advance\hsize by -21pt \divide\hsize by 2
    \let\pdfpagewidth=\pagewidth
    \let\pdfpageheight=\pageheight
    \pdfpagewidth=210truemm \pdfpageheight=297truemm % A4 Paper size
    \edef\pdfhorigin{\pdfvariable horigin}
    \pdfhorigin=\pdfpagewidth \advance\pdfhorigin by -\@hsize \divide\pdfhorigin by 2
    \let\homework=\ish@mew@rktrue
    \let\showanswer=\answershowntrue
    \parindent=0pt \baselineskip=13pt plus1ex minus1ex
    \lineskip=3pt \lineskiplimit=3pt
    \@beginexamtwocols
\else
    \def\nextq{\vfill\eject}
    % Constants (Include Boolean)
    \pagewidth=210truemm \pageheight=297truemm % A4 Paper size
    \edef\pdfhorigin{\pdfvariable horigin}
    \edef\pdfvorigin{\pdfvariable vorigin}
    \hsize=\pagewidth \vsize=\pageheight
    \advance\hsize by -4truecm
    \advance\vsize by -6truecm
    \pdfhorigin=2truecm \pdfvorigin=3truecm
    \parindent=0pt \baselineskip=13pt plus1ex minus1ex
    \lineskip=3pt \lineskiplimit=3pt
    \@beginexamonecol
\fi
}

\outer\def\endexam{%
\iftwocols
    \def\breakcolumn{\vfill\eject}
    \def\nextq{\vfill\eject\hbox to\hsize{}\vfill\eject}
    \let\bc\breakcolumn
    \@endexamtwocols
\else
    \def\nextq{\vfill\eject}
    \@endexamonecol
\fi
\openout\sc@rewrite=\jobname.tdt
\write\sc@rewrite{\directlua{tex.sprint(scorecnt)}}
\write\sc@rewrite{\the\pageno}
\closeout\sc@rewrite}

\def\@beginexamonecol{\mainfont
\nopagenumbers
\hbox{}
\vskip 0pt plus 1filll
\centerline{\titlefont\the\maintitle}
\vskip 0.5pc
\centerline{\subtitlefont\the\subtitle}
\vskip 2pc
\boxit{1pt}9{\hbox to0.95\hsize{\bsf 성명: \hss}}
\vskip 5pt
\boxit{1pt}9{%
    \hbox to\hsize{\hss\vbox{%
    \hbox to0.85\hsize{\hskip 0pt plus 1fil\vbox{\titleexplainfont
        \kern 1em
        \baselineskip=7mm plus 1em minus 1em
        \hbox{$\bullet$ 총점은 {\bsf\t@talsc@re\ignorespaces 점}이며, 제한시간은 {\bsf\the\timelimit 분}입니다.}
        \hbox{$\bullet$ 시험지는 표지 제외 총 {\bsf\t@talpagecnt 페이지}입니다.}
        \hbox{$\bullet$ 증명이 아닌 정답이 있는 문제는 정답을 {\bsf 알아볼 수 있게} 표시하시오.}
        \hbox{$\bullet$ 풀이과정은 각 문제 밑에 서술형으로 쓰시면 됩니다.}
        \hbox{$\bullet$ 풀이과정은 {\bsf 정자로} 쓰십시오. 그렇지 않을 경우, 불이익이 있을 수 있습니다.}
        \hbox{$\bullet$ 풀이과정을 쓸 때에 {\bsf 여백이 부족}할 경우, 감독관에게 여분의 종이를 요구할 수 있습니다.}
        \hbox{$\bullet$ {\bsf 컨닝} 혹은 {\bsf 분위기를 흐리는 자}는 성적이 {\bsf 0점}처리 됩니다.}
        \kern 1em
    }\hskip 0pt plus 3fil}}\hss}
}
\vskip 5pt
\begingroup
\spaceskip=1em plus 1ex minus 1ex
\boxit{1pt}9{%
\hbox to0.95\hsize{\hss\titlewarningfont \# 시험이 시작되기 전 까지 페이지를 넘기지 마시오 \#\hss}
}
\endgroup
\vskip 0pt plus 1filll
\eject
\hbox{}
\vfill
\eject
\advance\pageno by -2
\output={
    \footline={\bsf\hfil\folio\ / \t@talpagecnt\hfil}
    \ifnum\pageno=1
    \shipout\vbox{%
        \hbox{}
        \vskip-2cm
        \centerline{\titlefont\the\maintitle}
        \vskip 0.5pc
        \centerline{\subtitlefont\the\subtitle}
        \vskip 1pc
        \hrule height1.5pt
        \vskip 1pc plus -1fill
        \box255
        \makefootline
    }
    \else\ifodd\pageno
    \headline{\hfil\vbox{\hbox to\hsize{\the\maintitle\hfil\the\subtitle}\kern1ex\hrule height1pt}\hfil}
    \else
    \headline{\hfil\vbox{\hbox to\hsize{\the\subtitle\hfil\the\maintitle}\kern1ex\hrule height1pt}\hfil}
    \fi
    \shipout\vbox{\makeheadline\pagebody\makefootline}
    \fi
    \advancepageno
    \ifnum\outputpenalty>-20000\else\dosupereject\fi
}
}

\def\@beginexamtwocols{\mainfont
\ifish@mew@rk
\setbox0=\hbox to\@hsize{\headfont 성명 : \leaders\hrule\hskip2.5cm\hfil}
\else
\setbox0=\hbox to\@hsize{\headfont 성명 : \leaders\hrule\hskip2.5cm\hfil
\vbox{\halign{&\hfil\headfont##\cr 총점 : \t@talsc@re\unskip 점\cr 제한 시간 : \the\timelimit 분\cr}}}
\fi
\ht0=0pt
\setbox\titleb@x=\hbox to\@hsize{\hfil\vbox{\halign{&\hfil##\hfil\cr
\titlefont\the\maintitle\cr\headfont\the\subtitle\cr}}\hfil}
\setbox\headb@xeven=\hbox to\@hsize{{\headfont\the\maintitle}\hfil{\headfont\the\subtitle}}
\setbox\headb@xodd=\hbox to\@hsize{{\headfont\the\subtitle}\hfil{\headfont\the\maintitle}}

\output={\if L\leftright
\global\setbox\leftb@x=\box255
\global\let\leftright=R
\else
\shipout\vbox{\offinterlineskip
\ifnum\count0=1 \box\titleb@x\box0
\else\ifodd\count0\copy\headb@xodd\else\copy\headb@xeven\fi\fi
\kern0.65pc\hrule height1pt\kern0.65pc
\hbox to\@hsize{\hss\box\leftb@x\kern10pt\vrule width1pt\kern10pt\box255\hss}
\bigskip\hbox to\@hsize{\hss\pagenum\hss}\vss}
\global\let\leftright=L
\advancepageno\fi}}
\outer\def\answersheet{\if L\leftright
\vfill\breakcolumn\null\vfill\eject
\else
\vfill\eject\fi
\advance\count0 by -1
\output={}
}

\def\@endexamonecol{%
\nextq
\advance\pageno by -1
\output={\shipout\box255}
\ifodd\pageno\else\hbox{}\nextq\fi
\hbox{}
\vskip 0pt plus 1filll
\begingroup
\spaceskip=1em plus 1ex minus 1ex
\boxit{1pt}9{%
\hbox to0.95\hsize{\hss\titlewarningfont \# 시험이 시작되기 전 까지 페이지를 넘기지 마시오 \#\hss}
}
\endgroup
\vskip 0pt plus 1filll}
\def\@endexamtwocols{\relax}

\catcode`@=12
\message{Exam Class : Version 2.00}
